% Section 7: Conclusion

Short answer: yes---higher L2 adoption decongests Ethereum's fee market in the short run, but the relief is partial and local in time. A 10 percentage point increase in posting-clean adoption lowers L1 base fees by roughly 13\% (about 5~Gwei or \$0.14 for a 21k-gas transfer at the pre-Dencun mean), and deviations from the long-run relation decay with an 11-day half-life. Together with the dynamic profile in Figure~\ref{fig:lp_irf} and the ECM benchmark in Table~\ref{tab:c2_ecm}, these numbers provide regime-aware causal evidence that the rollup-centric roadmap already buys near-term congestion relief.

Conceptually, the paper introduces a posting-clean adoption measure that captures user migration rather than posting load, a demand factor that avoids mediator contamination, and a regime-aware ITS-ECM template for monitoring rollup-centric scaling. Substantively, it delivers the first cross-regime causal estimate of how aggregate L2 adoption decongests Ethereum's mainnet and translates the semi-elasticity into Gwei and dollar savings that are directly interpretable for protocol designers and users.

These claims are bounded. Inference is local to the pre-Dencun regime where adoption still moves, and precision fades beyond roughly a month of horizons. Instrument strength is modest, so simultaneity concerns are handled with cautious timing diagnostics rather than strong exclusion. As summarized in Section~\ref{sec:discuss:limitations}, these boundaries keep confirmatory claims narrow while flagging where additional variation is needed.

For protocol designers and governance bodies, the practical implication is that fee-market reforms and L2 ecosystem support should be evaluated jointly. Moving L2 user share from 60\% to 80\% would lower median base fees by roughly a quarter at pre-Dencun demand levels, putting adoption subsidies on the same order as the fee changes analyzed around the London upgrade \citep{LiuEtAl2022EIP1559}. In the blob era, incentives that shift activity onto rollups or smooth posting schedules operate alongside the blob-fee parameters in \citet{eip4844}, making adoption-driven interventions a complementary lever rather than a substitute for base-fee tuning. Future work should extend the confirmatory window as post-Dencun variance widens, seek quasi-experimental shocks in blob pricing or sequencer operations, and map distributional incidence using address-tagged data so that welfare gains from rollup-driven congestion relief can be allocated across user types.
