% Section 2: Related Work

\subsection{Fee-Market Design and Ethereum Upgrades}
\label{sec:literature:fee-market}

Scholarship on Ethereum's fee market shows how protocol upgrades reshape incentives without immediately expanding Layer-1 (L1) throughput. EIP-1559's base-fee burn and elastic block size improved price discovery and reduced fee volatility while leaving the hard cap on computation unchanged \citep{EthereumFoundation2021}. The Merge stabilized slot times and validator incentives without materially increasing execution capacity. Dencun's EIP-4844 then introduced dedicated blob space that dramatically reduced Layer-2 (L2) posting costs \citep{eip4844}.

Empirical analyses of EIP-1559 document how the new fee mechanism affects transaction fees, waiting times, and consensus margins \citep{LiuEtAl2022EIP1559}, while recent work on L2 arbitrage and rollup pricing studies cross-rollup spreads and the interaction between posting costs and liquidity provision \citep{GogolEtAl2024L2Arbitrage,WangCrapisMoallemi2025Posting}. Existing empirical work on Ethereum's fee market and rollups therefore either focuses on a single upgrade such as EIP-1559 or on protocol-level behavior inside specific rollup or application ecosystems, carefully quantifying local changes in fees, spreads, or posting costs but not the total effect of aggregate L2 user adoption on mainnet congestion across multiple protocol regimes. Industry observatories track the resulting growth of optimistic and zero-knowledge rollups, transitions from calldata to blob usage, and the emergence of posting-fee arbitrage,\footnote{For example, the L2Beat (\url{https://l2beat.com}) and Dune (\url{https://dune.com}) dashboards track Ethereum L2 total value locked, transaction volumes, posting costs, and blob usage; the specific snapshots used in this study were accessed in 2024 and are archived with the replication code.} but they typically treat L2 posting as part of user demand or abstract from macro shocks that jointly affect L1 congestion and L2 adoption. Our design fills this gap by treating L2 adoption as a continuous treatment and explicitly modeling the sequence of London, Merge, and Dencun regimes.

\subsection{Empirical Congestion and Causal Time-Series Methods}
\label{sec:literature:causal-ts}

Causal and time-series methods developed in adjacent technology and financial settings provide templates for credible evaluation of congestion policies. Interrupted time series (ITS) and segmented regression remain staples for policy impact analysis \citep{bernal2017,penfold2013}. Continuous-treatment event studies extend difference-in-differences logic to dosage-style shocks with explicit pre-trend tests \citep{deChaisemartinDHaultfoeuille2020}. Bayesian Structural Time Series (BSTS) constructs probabilistic counterfactual paths with state-space components for trends, seasonality, and contemporaneous covariates \citep{brodersen2015}, and Regression Discontinuity in Time (RDiT) exploits sharp policy boundaries when smoothness assumptions hold \citep{hausman2018}. These designs have been deployed in fintech launches, payment reforms, and energy-market interventions, and they underlie several recent empirical studies of blockchain fee dynamics and rollup pricing. Yet existing congestion studies rarely combine DAG-guided adjustment sets, mediator exclusion, and semi-elasticity reporting that maps coefficients into user-level cost changes.

\subsection{Broader Congestion and Market-Design Literatures}
\label{sec:literature:market-design}

Regulatory and market-microstructure literatures highlight the perils of conditioning on post-treatment variables when evaluating market design. Work on tax holidays, exchange-fee rebates, and telecom interconnection policies stresses the need for clean treatment definitions and transparent adjustment sets to maintain credibility when interventions unfold over multiple regimes. In the rollup-centric roadmap, L2 adoption both responds to and influences L1 congestion, so empirical strategies must avoid conditioning on posting flows and clearly distinguish exploratory diagnostics from confirmatory estimands.

Viewed through this lens, Ethereum's L1/L2 stack resembles other congestion-pricing problems in transportation networks, electricity grids, and payment systems: multiple service layers share a common bottleneck, and welfare depends on how incentives, fee schedules, and governance are coupled across layers. Existing studies either focus on single upgrades, rely on contemporaneous correlations pulled from dashboards, or embed L2 posting in both treatment and controls, diluting the estimand. To our knowledge, there is no regime-aware, DAG-grounded causal study that estimates the total effect of L2 adoption on L1 congestion across London, the Merge, and Dencun, nor one that pairs a posting-clean treatment with a demand factor that excludes mediator pathways. This study fills that gap by providing cross-regime semi-elasticities and adjustment dynamics that speak directly to Ethereum's rollup-centric scaling roadmap.
